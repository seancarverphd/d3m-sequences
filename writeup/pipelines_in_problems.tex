\documentclass{article}
\title{Pipeline Diversity within Problems}
\author{Data Machines Corporation}
\begin{document}
\maketitle

We analyze the diversity of pipelines within the Winter 2020
Evaluation of the D3M project, within problems, but across
performers.

This document consists of two parts.  The first part (comprising most
of the document) has 103 sections, one for each problem considered.
The second part (comprising the last several pages of the document)
contains a table of codes for primitives, and their descriptions.
These codes are used throughout the document.

Within each section, the problem considered is notated at the top of
the page.  There are two parts to each section: an image showing
``distances'' between primitives, and a table, showing primitives in a
multiple alignment.

The image was generated by computing the edit distances (Levenshtein
distance) between pipelines (number of insertion, substitutions,
deletions, etc, needed to bring one sequence to the other).  Each dot
in the scatter plot represents one pipeline evaluated for that
problem.  The color of the corresponding dot categorizes the 9
performers (tamu, nyu, sri, uncharted, etc).

The edit distances between pipelines were then passed to a spring
layout procedure (Fruchterman-Reingold), to determine the $x-$ and
$y-$ location of the pipelines on the two-dimensional plane of the
figure.  The axes of the plot are meaningless, but pipelines with
greater edit distances between them should appear further apart on the
graph.  Technically, we pass the reciprocal of the edit distance to
the F-R code, but the software does not accept infinity.  As a result,
identical pipelines appear a small distance apart.  That said, it is
often the case that scatter points overlap if and only if
corresponding pipelines are identical.

Note that there is some loss of information in moving the sequences to
two dimensions, as only three points can be represented equidistant in
the plane.  As a result, one can not conclude that points that are not
equidistant on the plot are not equidistant in edit space.

The image gives a rough idea of what is going on.  The table of
multiple sequence alignment, second part of each section, gives a more
precise picture.  The alignments were creating with a program called
mafft which was originally intended for biological sequences, by
minimizing an objective function penalizing gaps and substitutions.
There are a few artifacts here.  A primitive denoted ** indicates a bug
(rare).  Because gaps are penalized more than substitutions, the end
of sequences don't always align as they might be expected to.  Colors
of pipelines correspond to colors of scatter points (performers).

Primitive codes and their meanings are listed at the end of the document.


\end{document}
