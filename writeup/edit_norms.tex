\documentclass{article}
\title{Measuring Pipeline Diversity}
\author{Data Machines Corporation}

\begin{document}
\maketitle

If a pipeline is sequence of primitives, how do we quantify the
diversity of a collection of pipelines, in terms of its primitives?
We have to decide what we mean by diversity.  Is a collection of only
two pipelines very diverse if all their primitives are different?  How
about a collection of many pipelines that differ in only one
primitive---is that more or less diverse?

The use the Levenshtein edit distance between pairs of pipelines gives
us a number describing the difference (diversity) between the pairs of
pipelines, in terms of primitives.

What about a collection?  We have a different edit distance for each
of $N$ pairs in the collection.  We put these $N$ numbers into a
vector $v$ of length $N$---the quantity $N$ is related to the number
of pipelines, $n$, by $N = \frac{(n)(n-1)}{2}$.

As a first idea, we can take the sum or average of the components of
this vector.  The sum is simply the $L_1$ norm (see below) because all
components are non-negative.  We can generalize this to the $L_p$ norm
for any $p$.
$$L_p(v) = \left(|v_1|^p + |v_2|^p + \dots +
|v_N|^p\right)^\frac{1}{p}.$$ What about generalizing the averages?
$$M_p(v) = \frac{L_p(v)}{N}.$$ The quantities $L_p(v)$ and $M_p(v)$
are is defined for all $0 < p < \infty$ but $L_p(v)$ satisfies the
properties of a norm only for $p \geq 1$. Two limits can be considered
for $L$ and by extension $M$ (the second is not really a limit nor is
it a norm, but is often used).
$$L_\infty(v) = \max\left\{|v_1|, |v_2|, \dots, |v_N|\right\}$$
$$L_0(v) = \mbox{the number of non-zero elements of $v$}.$$

\newpage
\section{Synthetic Scenarios}

Now we test these measures on synthetic data.  Consider the following
scenarios.  All pipelines below have 10 unique primitives.
\begin{enumerate}
\item 
  Performer X submits only 2 pipelines---differing by 10 substitutions.
\item
  Performer Y submits 6 pipelines all differing from each other by 10
  substitutions.
\item
  Performer W submits 2 sets of 3 identical pipelines differing by 10
  substitutions between groups.
\item
  Performer Z submits 5 identical pipelines plus 1 additional pipeline
  differing from all the rest by 10 substitutions.
\item
  Performer U submits a sequence of pipelines differing from each
  other by a number of substitutions equal to the absolute value of the
  relative position in the sequence.
\item
  Performer V submits a sequence of pipelines all differing from each
  other by a single substitution.
\end{enumerate}

\section{Relative Orderings}

\begin{eqnarray}
  L_1 & : & Y > W > Z > U > V > X \\
  L_2 & : & Y > W > Z > U > X > V \\
  L_3 & : & Y > W > Z > X > U > V \\
  L_\infty & : & Y = W = Z = X > U > V \\
  L_0 & : & Y = U = V > W > Z > X \\
  M_1 & : & X = Y > W > Z > U > V \\
  M_2 & : & X > Y > W > Z > U > V \\
  M_3 & : & X > Y > W > Z > U > V \\
  M_\infty & : & X > Y = W = Z > U > V \\
  M_0 & : & X = Y = U = V > W > Z 
\end{eqnarray}

\section{Discussion}
Notice that all 10 measures agree that:
$$Y \geq W \geq Z \geq U \geq V$$
If you look back at the scenario, this ordering makes sense.

Now notice that the position of $X$ is very variable throughout the
measures.  What is different about X?  Notice that the collections $Y,
W, Z, U, \mbox{and } V$ all had six pipelines whereas $X$ had but two.
We can surmise that the differences in the measures appear largely
across collections of different sizes.

Note that the following orderings don't make sense:
$$X > Y$$
$$X > W$$
$$X > Z$$ Why?  Because all of $Y$, $W$, and $Z$ contain a pair of
distance 10 pipelines, plus additional pairs at nonzero distance
whereas X contains just one pair at distance 10.  Clearly $X$ should
be deemed less diverse than $Y$, $W$, and $Z$.  That said, all 5 of
the $M$ measures break at least one of these orderings, whereas none
of the $L$ measures do.

Of the 5 remaining measures $L_\infty$ and $L_0$ both lead to a lot of
equalities which may not be desirable.  The other three measures all
place $X$ in a different relative position to $U > V$.  The higher the
value of $p$, the more collections of fewer pipelines that are more
distant are deemed more diverse than collections with more pipelines
that are separated but less distant.

\section{Counterexample}
Because all same-sized collections chosen for the examples above have
a consistent ordering under the norms, it becomes natural to wonder if
same-sized collections are always ordered the same way regardless of
the norm used.

This property does \emph{not} hold as the following counterexample
shows:

Performer S submits $k$ identical pipelines plus one additional
pipeline at distance $d$.  There are $k+1$ pipelines and $k (k+1)/2$
pairs, of which $k$ pairs have distance $d$ and the rest have distance
0.  S is the same as $Z$ with $k = 5$ and $d = 10$.

Performer T submits $n$ pipelines and $N$ pairs, where $N = n (n-1) /
2$.  All of these pipelines have distance $e$ from each other.  $S$
and $T$ have the same numbers of pipelines/pairs if $n = k+1$.  T is
the same as V with $e = 1$, $n = 6$, and $N = 15$.

Both S and T have simple formulas for the measures, coming from their
simple structures.  The formulas are displayed in the following table:

\begin{tabular}{c|ccc}
  performer & $L_1$ & $L_2$ & $L_\infty$ \\
\hline
  S & $dk$ & $d \sqrt{k}$ & $d$ \\
  T & $eN$ & $e \sqrt{N}$ & $e$
\end{tabular}

Filling in the values of these measures for the parameters that define
$Z$ and $V$, we find

\begin{tabular}{c|ccc}
  performer & $L_1$ & $L_2$ & $L_\infty$ \\
\hline
  Z & $50$ & $22.36$ & $10$ \\
  V & $15$ & $3.87$ & $1$
\end{tabular}

This table reproduces the consistent ordering found above.

Now reparameterize so that $d = 10$, $k = 10$, $e = 2$, $n = 11$, and
$N = 55$. Then the values for these measures transform to:

\begin{tabular}{c|ccc}
  performer & $L_1$ & $L_2$ & $L_\infty$ \\
  \hline
  S & $100$ & $31.62$ & $10$ \\
  T & $110$ & $14.83$ & $2$
\end{tabular}

Note the measures $L_1$ and $L_2$ reversed the ordering of the
diversity of collections S and T.

\section{Conclusion}
Consistent with the story we saw above, higher values of $p$ deem
fewer larger edit distances as more diverse than a greater number of
smaller edit distances.  There is no one single measure of diversity.
Perhaps we can report $L_1$, $L_2$, and $L_\infty$ as three relevant
measures describing diversity.  All of these measures have natural
interpretations.  Many times, all three proposed measures will agree,
as it was hard to find the counterexample above.

If I had to pick one measure to use consistently, without the others,
it would be $L_2$ because it is a natural balance between the two
extremes.

\end{document}
